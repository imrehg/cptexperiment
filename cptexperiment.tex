\documentclass[10pt,a4paper]{report}
\usepackage[latin1]{inputenc}
\usepackage{amsmath}
\usepackage{amsfonts}
\usepackage{amssymb}
\begin{document}

Earlier experimental results include \cite{Brandt1997}.
Detailed discussion of the buffer gas' effect on the CPT signal can be found in \cite{Erhard2001}.
Detailed analytical discussion of the CPT signal (all the levels) in \cite{Taichenachev2003}.
Invited paper about spectroscopy with CPT \cite{Taichenachev2003} (predates frequency comb but includes narrowing).

People investigate Rb in similar (but CW) situation. They find very low detuning induced shifts and very narrow linewidths. \cite{Erhard2000}.

Older theory and experiment, using extra RF field \cite{Vanier1998}.

Anoter... \cite{Vanier1998}

Buffer-gas induced absorption resonance, that is narrower than then the transmission resonance \cite{Mikhailov2004}. Buffer-gas enhances nonlinear effects??? We must have loads of that! More in \cite{Mikhailov2004,Lukin1997,Harris1999,Johnsson2002}. Most of the details there are about coherently enhanced 4-wave mixing (read more!), and some time the optical medium is the dense one, not the buffer gas (which is in practice transparent to our fields).

\section{Frequency combs}

\section{Buffer gas topics}

Apparently the buffer-gas issues are not completely resolved. As fart back as 1999 people still had contradictory results how the high pressure of different gasses affect the CPT signal. E.g. \cite{Gozzini1999} has some experiments on the subject, but mostly cited for the references within.

\section{Calculation techniques}

Collection of calculation tricks used in the literature.
Although for CW, this is interesting how do they calculate the expected CPT lineshape \cite{Novikova2005}


\bibliographystyle{plain}
\bibliography{refs}
\end{document}
